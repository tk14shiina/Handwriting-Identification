\documentclass{article}
\usepackage[utf8]{inputenc}
\usepackage{graphicx}
\usepackage{wrapfig}
\usepackage{multirow}
\usepackage{tabu}
\usepackage{array}
\usepackage{abstract}

\title{Handwriting Identification}
\author{Le Kieu Anh, Pham Quoc Trung, Vu Le Mai}
\date{July 2019}

\begin{document}

\maketitle

\begin{abstract}
After a week of learning and working, our group decided to build a \textbf{Handwriting Identification} model. In this paper, we introduced 2 general concepts of Machine Learning: \textbf{The Unified Framework} and \textbf{TEFPA}, discussed how the each concept was applied in our model, and described the Handwriting Identification model in detail.
\end{abstract}

\tableofcontents
\newpage

\section{Introduction to Machine Learning:}
\subsection{TEFPA:}
\textbf{TEFPA} is the general model in Machine Learning, including 
\begin{itemize}
    \item T (Task) - input and output of the model
    \item E (Experience) - dataset used in the model
    \item F (Function) - functions used in the model
    \item P (Performance) - assessment of the model (loss/accuracy)
    \item A (Algorithm) - algorithms applied in the model
\end{itemize}

\subsection{A Simplified yet Unified Framework:}
A framework that is used to solve all problems with machine learning.\\

\begin{figure}[h!]
\centering\includegraphics[scale=0.4]{UnifiedFramework.jpg}
\\Image 1. Machine Learning Unified framework\\
\end{figure}\

\underline {\textbf{Notes:}} \\
1. $g_1$: Basis function (linear function)\\
2. $Z_g, Z_h$: Embeddings/ Coordinate vector / Basis vector \\
3. $g_2$: Basis function \\
4. $Z_x, Z_y$: comparable vectors\\

\underline{\textbf{Brief Description:}} \\
- Input goes through basis function $g_1$ to extract important features as embeddings (coordinate vectorS). \\
- Function $g_2$ projects embeddings into the same space to compare predicted output with the real output.

\section{Handwriting Identification}

Handwriting Identification is a program that identifies writer from an image of handwritten sentences. \\

\subsection{Handwriting Identification in Unified Framework}

\begin{figure}[h!]
\centering\includegraphics[scale=0.5]{handwritingUF.jpg}
\\Image 2. Handwriting Identification's Unified framework\\
\end{figure}\

\underline{\textbf{Brief Description:}} \\
\begin{itemize}
    \item Original Model: \\
    + Image of handwriting goes through CNN to extract important features as embeddings (coordinate vectors).\\
    + Softmax function generates embeddings into probability vectors, using argmax to find writer that has the highest probability. 
    \item Modified model: \\
    + Image of handwriting goes through CNN to extract important features as embeddings (coordinate vectors).\\
    + Embedding goes through kNN to find  writer with the nearest coordinate vector to it in the vector space.
\end{itemize}

\subsection{Handwriting Identification's Task:}

- Input: Image of handwritten sentences 
\begin{figure}[h!]
\centering\includegraphics[scale=0.3]{input.png}
\\Image 3. Input image\\
\end{figure}\

- Output: Name of writer
\begin{figure}[h!]
\centering\includegraphics[scale=0.6]{output.png}
\\Image 4. Output\\
\end{figure}\

- Purpose: To identify writer based on images of handwriting. \\

\subsection{Handwriting Identification's Experience:}
- Dataset: IAM Handwriting Database \\
- Description: The dataset includes 657 writers contributed samples of their handwriting, which were scanned at a resolution of 300dpi and saved as PNG images with 256 gray levels. \\
- Each image is cropped into 113x113-size images, to increase the capability of the database.\\
\textbf{- The dataset is split into train files, test files, and validation files with the ratio 4:1:1} \\ 
- We imported images of handwritten sentences of our class's students as new data and used 'convert(mode= "L")' to convert images into grayscale.

\subsection{Handwriting Identification's Function:}

\textbf{Structure:} Consists of CNN (Convolutional Neural Network)\\
\\
\\
\\
\\
\\
\\
\begin{figure}[h!]
\includegraphics[scale=0.35]{CNN.jpg}
\centering Image 5. CNN in Handwriting Identification\\
\end{figure}
\\
CNN was used for the machine to extract and learn the important features in the input image.\\ 
CNN consists of 3 main types of layers: Convolution Layer, Pooling Layer and Fully-connected Layer:\\
+ \textbf{Convolution Layer}: the objective of convolution is to extract high-level features of the image from the input image. The first convolution layer is responsible for extracting low-level features. As there are more convolution layers, the machine are able to learn higher-level features of the image. \\
+ \textbf{Pooling Layer}: is used to reduce the size of the convolved feature in the image, in order to reduce the computationtal power required to run the process. \\
+ \textbf{Fully-connected Layer}: performs classification based on the features extracted by previous layers. It typically contains softmax activation function in order to output a probability vector for each of the classification labels. \\

Our original model classified handwriting of 50 writers. In order to change to a recognition model, we changed the structure of the model by removing the last layer (Fully-Connected Layer) in CNN and adding kNN algorithm. \\

\subsection{Handwriting Identification's Performance:}
To measure the loss in the session we use \textbf{cross entropy} to measure the differences between the training datas (cropped images) and the samples (images from dataset)\\

\subsection{Handwriting Identification's Algorithm:}
-Optimizer : ADAM \\
-kNN (k Nearest Neighbors) : a supervised-learning that use K nearest data in the dataset to label new input data. We use this to calculate global average weights and label unseen handwritings (following major voting). 

\section{Resources}
http://www.fki.inf.unibe.ch/databases/iam-handwriting-database

\end{document}


